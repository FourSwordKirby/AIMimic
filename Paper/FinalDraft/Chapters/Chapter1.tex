% Chapter 1

\chapter{Introduction} % Main chapter title

\label{Chapter1} % For referencing the chapter elsewhere, use \ref{Chapter1} 

%----------------------------------------------------------------------------------------

% Define some commands to keep the formatting separated from the content 
\newcommand{\keyword}[1]{\textbf{#1}}
\newcommand{\tabhead}[1]{\textbf{#1}}
\newcommand{\code}[1]{\texttt{#1}}
\newcommand{\file}[1]{\texttt{\bfseries#1}}
\newcommand{\option}[1]{\texttt{\itshape#1}}

%----------------------------------------------------------------------------------------


Fighting games are unique among competitive multiplayer games in that they are real-time, 1-on-1 contests where small mistakes lead to huge consequences. The best way to get better at these kinds of games is to practice against other humans, but that is not always possible. While the option to play online exists, it is not ideal due to the lag introduced by network latency. In addition, the AI in these games are generally considered a poor substitute for real players. They often exploit natural advantages such as perfect reaction time and perfect game state information, but even disregarding that they still only have fixed spectrum of behavior patterns which players can learn to exploit and consistently defeat. Worse still is that these behavior patterns might not even be representative of the human opponents that players encounter in competition.

That said, there are avenues to improve the AI in fighting games to make them useful for players. One approach is to make an optimal AI which is able to adapt its strategy based on its performance. This would provide players a challenge by removing the ability to exploit the AI, but it still doesn't necessarily capture the strategies and techniques used by other human players. Another approach is to make a \textit{human-like} AI, one that plays like another specific human opponent. This task seems feasible, as long-time fighting game players can identify the differences between the playstyles of different players, meaning that there is some quality that differentiates one behavior from another.

In this research, we investigate planning-based approaches to creating human-like AI. To do this, we first explore previous approaches taken to create human-like AI and discuss their merits and limitations. We also describe other attempts at creating AI for fighting games to contextualize our efforts compared to theirs. We then introduce the environment we created to test our approach and define concepts and terminology used by our algorithm. Then, we describe our algorithm, where we plan on the actions provided by human demonstrations to reach a desired outcome. Lastly, we test our algorithm and compare its performance to other existing implementations.

%----------------------------------------------------------------------------------------
